\documentclass[a4paper, 11pt]{article}
% For UTF-8 encoding
\usepackage[utf8]{inputenc}
% For proper hyphenation etc.
\usepackage{babel}
% For clickable links
\usepackage{hyperref}
% Set page margins
\usepackage[top=2cm, bottom=2cm, left=2.5cm, right=2.5cm]{geometry}
\usepackage{amssymb}
\usepackage{amsmath}
\usepackage{tikz}
\usepackage{pgfplots}
\usetikzlibrary{calc,shadings}
\usetikzlibrary{positioning}
\usepackage{amsthm}
\usepackage{color}
\usepackage{algorithm,algorithmic}

%Definitions
\newtheorem{mydef}{Definition}
\newtheorem{anmerkung}{Anmerkung}
\newtheorem{beispiel}{Beispiel}
\newtheorem{lemma}{Lemma}
\newtheorem{theorem}{Theorem}
\newtheorem{corollary}{Korollar}
\newtheorem{satz}{Satz}

\makeatletter
\renewenvironment{quotation}
{\list{}{\listparindent=1.5em
		\itemindent=0pt
		\parsep\z@ \@plus\p@}%
	\item\relax}
{\endlist}
\makeatother

\newenvironment{customlegend}[1][]{%
	\begingroup
	% inits/clears the lists (which might be populated from previous
	% axes):
	\csname pgfplots@init@cleared@structures\endcsname
	\pgfplotsset{#1}%
}{%
	% draws the legend:
	\csname pgfplots@createlegend\endcsname
	\endgroup
}%

%definitions
\def\addlegendimage{\csname pgfplots@addlegendimage\endcsname}
% definition to insert numbers
\pgfkeys{/pgfplots/number in legend/.style={%
		/pgfplots/legend image code/.code={%
			\node at (0.295,-0.0225){#1};
		},%
	},
}


%opening
%\title{	Proseminar \glqq Theoretische Informatik\grqq{}\\
%		Lineare Optimierungsprobleme und deren Lösung mit Hilfe des \textsc{Simplex}-Verfahrens}
%\author{Tilman Hinnerichs}
%\date{Sommersemester 2018}

\usepackage{fancyhdr}

\newcommand{\HRule}[1]{\rule{\linewidth}{#1}}
\setcounter{tocdepth}{5}
\setcounter{secnumdepth}{5}

%-------------------------------------------------------------------------------
% TITLE PAGE
%-------------------------------------------------------------------------------

\begin{document}
	
	\title{ \normalsize \textsc{Seminar: Selected Topics in Logic and Verification }
		\\ [2.0cm]
		\HRule{0.5pt} \\
		\LARGE \textbf{\uppercase{A Summary on:\\
			A Pivoting Algorithm for Convex Hulls and Vertex Enumeration of Arrangements and Polyhedra}
		\HRule{1pt} \\ [0.5cm]
		\normalsize \today \vspace*{5\baselineskip}}
	
	\date{Summer semester 2020}
	
	\author{
		Tilman Hinnerichs \\
		Matrikelnummer: 4643427 \\ 
		Technische Universität Dresden\vspace{1cm}\\
		Tutor: Dr. Florian Funke }}
\maketitle
\newpage
\begin{abstract}
	Test
\end{abstract}

\tableofcontents

\newpage

\begin{itemize}
	\item What audience? students with simple to no prior knowledge
	\item Don't just formulate presentation in words $\rightarrow$ add more stuff from paper
	\item add facet enumeration
	\item future work $\rightarrow$ paper citing this paper
	\item rewrite their definitions with yours for extra points
	\item add some of the theorems from the Simplex summary
\end{itemize}

\section{Polyhedra and Arrangements}

\section{The Vertex Enumeration Problem}
Definition of problem\\
introduction by example
\subsection{Duality to the Facet enumeration problem}
see what is written in Avis paper and copy it

\subsection{Types of approaches}
Motzkin vs Pivot based methods

\section{Simplex algorithm}
\subsection{Linear programs}
\subsection{The \textsc{Simplex}-Algorithm}
\begin{itemize}
	\item what does is do?
	\item How does it work
	\item why does it work? $\rightarrow$ translate some stuff from last presentation
\end{itemize}

\section{Avis and Fukudas Algorithm}
\subsection{How to move up the tree?}
Bland's rule and Criss-Cross rule
\subsection{What is it even doing?}
\subsection{Degeneracy}
with visual example from presentation
\subsection{Why is that good?}
Complexity

\section{Future Work based on this }
%body

\newpage

\begin{thebibliography}{9}
	\bibitem{introtoAlg} 
	Thomas H. Cormen, Charles E. Leiserson, Ronald L. Rivest und Clifford Stein.
	\textit{Introduction to Algorithms}. Third Edition. The MIT Press, 2009.
\end{thebibliography}


\end{document}